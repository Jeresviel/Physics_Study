\documentclass[11pt,twoside,a4paper,twocolumn]{article}
\usepackage[utf8]{inputenc}
\usepackage{amsmath}%math library
\usepackage{amssymb}
%\usepackage[german]{babel}
\usepackage[T1]{fontenc}
\usepackage{amsfonts}
\usepackage{makeidx}
\usepackage{graphicx}
\usepackage{geometry}
 \geometry{
 a4paper,
 total={170mm,257mm},
 left=20mm,
 top=20mm,
 }
\title{Masterarbeit}
\author{Jeremy WANG}
\date{\today}



\begin{document}

\maketitle
\tableofcontents
\section{Introduction}%first comment
\subsection{Ferroelectric}
\subsubsection{$BaTiO_3$}
\paragraph{thin film}

\section{physics}newton second law is $F=ma$.
The Newton's second law is $$F=ma$$.
The Newton's second law is \[F=ma\]
Greek Letters $\eta$ and $\mu$
Fraction $\frac{a}{b}$
Power $a^b$
Subscript $a_b$
Derivate $\frac{\partial y}{\partial t} $
Vector $\vec{n}$
Bold $\mathbf{n}$
To time differential $\dot{F}$
Matrix (lcr here means left, center or right for each column)
\[
\left[
\begin{array}{lcr}
a1 & b22 & c333 \\
d444 & e555555 & f6
\end{array}
\right]
\]
Equations(here \& is the symbol for aligning different rows)
\begin{align}
a+b&=c\\
d&=e+f+g
\end{align}



\end{document}