\documentclass[UTF8,a4paper,10pt]{ctexart}
\usepackage{ctex}
\usepackage{fontspec}
\usepackage{multicol}
\usepackage{graphicx}
\setmainfont{CMU Serif}
\setlength{\parindent}{2em}
\newcommand{\cpicn}[3]
{
\begin{figure}[h!]
\cpic{#1}{#2}
\caption{#3\label{#2}}
\end{figure}
}
\CTEXsetup[format={\Large\bfseries}]{section}
\title{\textbf{引力弹弓的原理及其应用}}
\author{\textit{王睿杰} 2017141475035}
\date{\today}
\begin{document}
\begin{multicols}{2}
[\maketitle
\paragraph{摘\quad 要}    \textit{刘慈欣在《流浪地球》中指出科学家计算出错,实际上太阳会加速老化从而爆发氦闪吞噬地球。但实际上即使太阳不加速老化仍会在10亿年内将地球上的液态水蒸发,50亿年后会变为膨胀为红巨星并吞噬地球,因此我们应当利用引力弹弓效应将地球移出现轨道,本文计算了木星的引力弹弓效应,证实电影中的方法是可以实现的。}
\paragraph{关键词}  \textit{流浪地球\quad 引力弹弓\quad 动量定理}]
\section{绪论}
    最近的科幻电影《流浪地球》十分热门,相信不少人都看过地球经过木星时利用木星引力加速的剧情
    在轨道动力学与航空航天工程中,引力弹弓(也被称为重力助推或绕行星变轨)是利用行星或其他天体的相对运动和引力效应改变飞行器的轨道和速度,以此来节省燃料、时间和计划成本。引力弹弓既可用于加速飞行器,也能用于减速飞行器。\par
    最初提出引力弹弓效应的是苏联科学家尤里·康德拉图克(Юрий Кондратюк),他在所署时间为“1918-1919”的论文“Тем кто будет читать, чтобы строить”(《致有志于建造星际火箭而阅读此文者》)中提出在两颗行星间飞行的飞船可以使用两行星卫星的重力实现轨道初段的加速和轨道末段的减速。弗里德里希·灿德尔(Friedrich Zander)在其1925年的论文“Проблема полета при помощи реактивных аппаратов: межпланетные полеты”(《星际飞行中喷气推进的问题》)中也提出了类似的构想。但是两者都未能意识到行星沿飞行器轨道施加的重力助推能够推进飞行器从而减少飞行器星际间飞行的燃料消耗,这一设想由迈克尔·米诺维奇(Michael Minovitch)于1961年提出。1959年,重力推进法得到了首次应用,当时苏联的探测器月球3号使用该法运行至月球背面并拍摄了该区域的照片,当时这一操作流程由克尔德什应用数学研究所所设计。\par
    在太阳系中,由于飞往内行星的飞行器的轨道方向是朝向太阳的,所以其可以获得加速度;而飞往外行星的飞行器由于是背向太阳飞行的,故其速度会逐渐降低。\par
    虽然内行星的轨道运行速度要比地球的快得多,但是飞往内行星的飞行器由于受到太阳引力作用而获得加速,其最终速度仍远高于目标行星的轨道运行速度。如果飞行器只是计划飞掠该内行星,就没有必要为飞行器降速。但是如果飞行器需要进入环该内行星的轨道,那么就必须通过某种机制为飞行器降速。\par
    同样的道理,虽然外行星的轨道运行速度要低于地球,但是前往外行星的飞行器在受到太阳引力作用而逐渐减速之后,其最终速度将仍低于外行星的轨道运行速度。所以也必须通过某种机制为飞行器加速。同时,为飞行器加速还能够减少飞行所耗时间。\par
    使用火箭助推是为飞行器加减速的重要方法之一。但是火箭助推需要燃料,燃料具有重量,而即使是增加很少量的负载也必须考虑使用更大的火箭引擎将飞行器发射出地球。因为火箭引擎的抬升效果不仅要考虑所增加负载的重量,也必须考虑助推这部分增加的负载质量所需的燃料的重量。故而火箭的抬升功率必须随着负载重量的增加而呈指数增加。\par
    而使用重力助推法,则飞行器无需携带额外的燃料就可实现加减速。此外,条件适宜的情况下,大气制动也可用来实现飞行器的减速。如果可能,两种方法可以结合起来使用,以最大程度的节省燃料。\par
    例如,在信使号计划中,科学家们即试用了重力助推法为这艘前往水星的飞行器进行减速,不过由于水星基本上不存在大气,所以无法使用大气制动来为飞行器减速。\par
    而飞往离地球最近的行星——火星和金星——的飞行器一般使用赫曼转移轨道法,该轨道呈椭圆形,其开始一端与地球相切,末尾一端与目标行星相切。该方法所消耗的燃料得到了尽可能的缩减,但是速度较慢——使用该方法的飞行器从地球达到火星需要1年多的时间(模糊轨道法使用的燃料更少,而速度则更慢)。\par
    如果使用赫曼转移轨道法前往外行星(木星、土星和天王星等),途中可能就要消耗掉数十年的时间,所需的燃料仍然很多,因为飞行器的航程长达8亿公里,同时还要抵抗太阳的引力。而重力助推则提供了一个无需附加燃料即可为飞行器加速的方法。所有飞往外行星的飞行器都使用了该方法。\par
    到目前为止利用过引力弹弓效应的飞行器有 “ 小行星3753"、 “ 水手10号 ” 、“旅行者1 号 ” 、"伽利略号 ” 、“新视野号 ”等等。\cite{wiki}\par
\section{原理}
    对于最简单的情况:两天体可视为质点,相互作用时机械能守恒,且速度方向平行,这种情况是完全弹性碰撞的正碰。设它们的质量和速度分别为$m_1,m_2,v_1,v_2$由动量守恒定律和机械能守恒定律可以推出碰撞后的速度:
    \begin{equation}
        v'_1 = \frac{(m_1-m_2)v_1 + 2m_2v_2}{m_1+m_2}
    \end{equation}
    \begin{equation}
        v'_2 = \frac{(m_2-m_1)v_2 + 2m_1v_1}{m_1+m_2}
    \end{equation}
    式中$v_1^{'},v_2^{'}$分别为碰撞后的速度。\par
    当天体之间质量差距非常大时,即$\frac{m_1}{m_2} \to 0$,因此上式变为
    \begin{equation}
        v'_1 = v_1 + 2v_2
    \end{equation}
    \begin{equation}
        v'_2 = v_2
    \end{equation}
    可以看出质量较大的天体速度不变,但质量较小的天体获得了$2v_2$的速度。\par
    真实的太阳系要复杂得多,很多时候并不能实现正碰撞,这时候速度矢量会变成二维或三维。由于绝大多数行星的黄道平面几乎共面,在太阳系内部发射探测器实际上可以近似看成平面问题。\par
    
\end{multicols}
\bibliographystyle{IEEEtran}
\bibliography{slingsshot_effect}
\end{document}