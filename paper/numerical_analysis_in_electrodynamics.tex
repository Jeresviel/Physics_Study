\documentclass[UTF8,a4paper,10pt]{ctexart}
\usepackage{ctex}
\usepackage{amsmath}
\usepackage{fontspec}
\usepackage{multicol}
\usepackage{newtxtext,newtxmath}
\usepackage{geometry}
\usepackage{bm}
\usepackage{listings}
\setlength{\parindent}{2em}
\geometry{a4paper,left=2.5cm,right=2.5cm,top=2.5cm,bottom=2.5cm}
\CTEXsetup[format={\Large\bfseries}]{section}
\title{\textbf{电动力学课程数值计算报告}}
\author{\textit{王睿杰} 2017141475035}
\date{\today}
\begin{document}
\begin{multicols}{2}
    [\maketitle]
    \section{绪论}
        目前物理大致分为三大部分:理论、实验和计算。对于普通本科生,主要以理论课为主,实验课为辅,而计算除了专设一门计算物理课程外,通常是通过伴随课程布置数值计算作业来学习。\par
        数值分析是分析的一个重要分支,指在数学分析问题中对数值近似(相对于一般化的符号运算)算法的研究。巴比伦泥板YBC 7289是关于数值分析的最早数学作品之一,它给出了$\sqrt {2}$在六十进制下的一个数值逼近,$\sqrt {2}$是一个边长为1的正方形的对角线,在公元前1800年巴比伦人也已在巴比伦泥板上计算勾股数$(3,4,5)$,即直角三角形的三边长比。\par
        数值分析延续了实务上数学计算的传统。巴比伦人利用巴比伦泥板计算 $\sqrt {2}$的近似值,而不是精确值。在许多实务的问题中,精确值往往无法求得,或是无法用有理数表示(如$\sqrt {2}$)。数值分析的目的不在求出正确的答案,而是在其误差在一合理范围的条件下找到近似解。\par
        在所有工程及科学的领域中都会用到数值分析。像天体力学研究中会用到常微分方程,电动力学中会使用偏微分方程组,最优化会用在资产组合管理中,数值线性代数是资料分析中重要的一部分,而随机微分方程及马尔可夫链是在医药或生物学中生物细胞模拟的基础。\par
        在计算机发明之前,数值分析主要是依靠大型的函数表及人工的内插法,十九世纪英国数学家查尔斯·巴贝奇曾经设计过用于计算的机械计算机——差分机与分析机,但因工艺所限没有完全实现,二十世纪电子计算机被发明,被大量运用于数值计算。计算机的内插算法仍然是数值分析软件中重要的一部分。\par
    \section{原理}
        \subsection{电动力学原理}
        微积分在物理学中有十分重要的作用,我们可以用微积分语言来描述物理学定律,假设我们已经有了关于微积分的基本知识,在电动力学中,有几个基本的实验定律:\par
        电磁理论最基本的实验定律是电荷守恒定律,电荷是粒子的基本属性,无论发生什么过程,一个系统的总电荷严格保持不变。电荷守恒定律在数学上用电荷连续性方程表示,设任意区域$V$,其边界为闭合曲面$S$,则有
        \begin{equation}
            \oint_S\bm{J}\cdot {\rm d}\bm{S} = -\int_V\frac{\partial\rho}{\partial t}{\rm d}V
        \end{equation}
        这是电荷守恒定律的积分形式,应用高斯公式
        \begin{equation*}
            \oint_S\bm{J}\cdot {\rm d}\bm{S} = \int_V\nabla\cdot\bm{J}{\rm d}V
        \end{equation*}
        即可得到微分形式
        \begin{equation}\label{continuity equation}
            \nabla\cdot\bm{J} + \frac{\partial\rho}{\partial t}= 0
        \end{equation}
        式中$\nabla = (\frac{\partial}{\partial x},\frac{\partial}{\partial y},\frac{\partial}{\partial z})$为矢量微分算符。上式被称为电流连续性方程,是电荷守恒定律的微分形式。\par
        库仑定律也是静电问题的基本实验定律,表述为:真空中静止点电荷$Q_1$对另一个点电荷$Q_2$的作用力$\bm{F}$为
        \begin{equation}\label{Coulomb_1}
            \bm{F} = \frac{Q_1Q_2}{4\pi\varepsilon_0r^3}\bm{r}
        \end{equation}
        式中$\bm{r}$为从$Q_1$指向$Q_2$的径矢,$\varepsilon_0$为真空电容率也就是真空介电常数。\par
        库仑定律仅给出了点电荷间作用力的大小与方向,并没有对原理进行解释,经过后来的研究表明电荷之间通过电场相互作用,点电荷所受力$\bm{F}$等于电荷量$q$和该点电场$\bm{E}$的乘积
        \begin{equation}
            \bm{F} = q\bm{E}
        \end{equation}
        由库仑定律\eqref{Coulomb_1}式可得到静止点电荷$Q$激发的电场强度为
        \begin{equation}
            \bm{E} = \frac{Q}{4\pi\varepsilon_0r^3}\bm{r}
        \end{equation}
        由实验得知电场满足叠加原理,如果空间中存在多个点电荷则一点电场强度等于每个电荷单独存在时产生的电场的和
        \begin{equation}
            \bm{E} = \sum_i\frac{Q_i}{4\pi\varepsilon_0r_i^3}\bm{r_i}
        \end{equation}
        在许多时候电荷连续分布在某一区域,取该区域内一体积元${\rm d}V$,该体积元的电荷满足
        \begin{equation*}
            {\rm d}Q = \rho{\rm d}V
        \end{equation*}
        式中$\rho$为电荷密度。\par
        根据上式,对整个电荷区域进行积分则得到空间中一点的电场为
        \begin{equation}\label{Coulomb_2}
            \bm{E}=\frac{1}{4\pi\varepsilon_0}\int_V\frac{\rho\bm{r}}{r^3}{\rm d}V
        \end{equation}
        按照理查德·费恩曼所说,给定了电荷,电场只需要通过计算这一项复杂的三重积分的工作就能得到,“严格说来,是一项计算机的工作!”。\par
        定义穿过闭合曲面$S$的电场$\bm{E}$的通量为面积分
        \begin{equation*}
            \oint_S\bm{E}\cdot{\rm d}\bm{S}
        \end{equation*}
        式中${\rm d}\bm{S}$为面元。可以由库仑定律推导出
        \begin{equation}
            \oint_S\bm{E}\cdot{\rm d}\bm{S} = \frac{1}{\varepsilon_0}\int_V\rho{\rm d}V
        \end{equation}
        上式被称为高斯定律或高斯定理。\par
        安培力定律是描述恒定电流相互作用的实验定律,可以仿照对库仑定律的操作,定义电流元$I{\rm d}\bm{l}$在磁场中所受到的力为
        \begin{equation}
            \bm{F}=I{\rm d}\bm{l}\times\bm{B}
        \end{equation}
        $\bm{B}$为该点的磁场。\par
        静磁学中另一个基本实验定律是毕奥-萨伐尔定律,设$\bm{J}$为源点电流密度
        \begin{equation}
            \bm{B}=\frac{\mu_0}{4\pi}\int_V\frac{\bm{J}\times\bm{r}}{r^3}dV
        \end{equation}
        同样我们可以定义沿闭合曲线$L$的磁场$\bm{B}$的环流量为线积分
        \begin{equation*}
            \oint_L\bm{B}\cdot{\rm d}\bm{l}
        \end{equation*}
        式中${\rm d}l$为线元,并且推导出
        \begin{equation}
            \oint_L\bm{B}\cdot{\rm d}\bm{l} = \mu_0\int_S\bm{J}\cdot{\rm d}S
        \end{equation}
        上式被称为安培定律或者安培环路定理。\par
        也可以计算出点电荷的电场的环路积分为零,因此
        \begin{equation}
            \oint_L\bm{E}\cdot{\rm d}\bm{l} = 0
        \end{equation}
        相对应地有被称为高斯磁定律的
        \begin{equation}
            \oint_S\bm{B}\cdot{\rm d}\bm{S} = 0
        \end{equation}
        上式说明不存在像电荷一样可以激发磁场的磁单极,目前没有任何实验观察到磁单极的存在。\par
        根据以上定律可以总结出静电磁学的麦克斯韦方程组
        \begin{equation}
            \left\{\begin{array}{l}
                \oint_S\bm{E}\cdot{\rm d}\bm{S} = \frac{1}{\varepsilon_0}\int_V\rho{\rm d}V\\
                \oint_L\bm{E}\cdot{\rm d}\bm{l} = 0\\
                \oint_L\bm{B}\cdot{\rm d}\bm{l} = \mu_0\int_S\bm{J}\cdot{\rm d}S\\
                \oint_S\bm{B}\cdot{\rm d}\bm{S} = 0
            \end{array}\right.
        \end{equation}
        或者利用散度和旋度,改写成微分形式
        \begin{equation}
            \left\{\begin{array}{l}
                \nabla\cdot\bm{E} = \frac{\rho}{\varepsilon_0}\\
                \nabla\times\bm{E} = 0\\
                \nabla\times\bm{B} = \mu_0\bm{J}\\
                \nabla\cdot\bm{B} = 0
            \end{array}\right.
        \end{equation}
        由于电场旋度恒为零,我们可以知道电场可以表示为一个标量场的梯度的形式
        \begin{equation}
            \bm{E} = -\nabla\varphi
        \end{equation}
        式中$\varphi$被定义为电势,由电荷在电场中做功
        \begin{equation}
            W = -q\int_a^b\bm{E}\cdot{\rm d}l
        \end{equation}
        $q$的选取是任意的,与场本身的性质无关,因此得到
        \begin{equation}
            \varphi(b) - \varphi(a) = -\int_a^b\bm{E}\cdot{\rm d}l
        \end{equation}
        可以知道电荷在电场内$a$点移动$b$点到所做功与路径无关而只与两点电势差有关,因此若没有确定的零电势,一点的电势没有什么意义,两点的电势差才具有物理意义。\par
        静电场与静磁场是相互独立的两种场,但是在电磁场发生变化时,电场与磁场会产生联系。\par
        法拉第定律是法拉第研究电磁感应时发现的实验定律
        \begin{equation}
            \oint_L\bm{E}\cdot{\rm d}\bm{l} = -\frac{{\rm d}}{{\rm d}t}\oint_S\bm{B}\cdot{\rm d}\bm{S}
        \end{equation}
        若环路$L$为空间中的一条固定回路,则上式为
        \begin{equation}
            \oint_L\bm{E}\cdot{\rm d}\bm{l} = \oint_S\frac{\partial\bm{B}}{\partial t}\cdot{\rm d}\bm{S}
        \end{equation}
        对于安培定律,从数学分析可得到
        \begin{equation}
            \nabla\cdot\nabla\times\bm{B} = \nabla\cdot\mu\bm{J} = 0
        \end{equation}
        恒定电流是闭合的,因此有
        \begin{equation}
            \nabla\cdot\mu_0\bm{J} = 0
        \end{equation}
        但在一般情况下,我们只能得到\eqref{continuity equation}式,因此安培定律仅适用于静磁学。麦克斯韦在研究时引入了被称为位移电流的新物理量$\bm{J}_D$,与$\bm{J}$构成一个闭合回路,得到新表达式
        \begin{equation}
            \nabla\times\bm{B} = \mu_0(\bm{J}+\bm{J}_D)
        \end{equation}
        由电流连续性方程和高斯定律可以得到$\bm{J}_D$最简单的表达式
        \begin{equation}
            \bm{J}_D = \varepsilon_0\frac{\partial\bm{E}}{\partial t}
        \end{equation}
        加入法拉第定律和麦克斯韦修改后的麦克斯韦-安培定律后就可以得到电动力学最重要的麦克斯韦方程组,由四个偏微分方程组成:
        \begin{equation}\label{Maxwell's equations}
            \left\{\begin{array}{l}
                \nabla\times\bm{E} = -\frac{\partial\bm{B}}{\partial t}\\
                \nabla\times\bm{B} = \mu\bm{J}+\mu\varepsilon\frac{\partial\bm{E}}{\partial t}\\
                \nabla\cdot\bm{E} = \frac{\rho}{\varepsilon}\\
                \nabla\cdot\bm{B} = 0
            \end{array}\right.
        \end{equation}
        $\varepsilon,\mu$为介电常数和磁导率。在真空中介电常数和磁导率表示为$\varepsilon_0$和$\mu_0$,在介质中有:
        \begin{equation}
            \begin{array}{c}
                \bm{D} = \varepsilon\bm{E}\\
                \bm{B} = \mu\bm{H}
            \end{array}
        \end{equation}
        麦克斯韦方程组主要描述了电磁场的性质,而动力学方程主要是牛顿运动定律
        \begin{equation}
            \bm{F} = \frac{{\rm d}\bm{p}}{{\rm d}t}
        \end{equation}
        式中$\bm{p}$为质点的动量,即牛顿定义质点所受力为质点动量的变化率。\par
        在经典力学的情况下,质量与速度无关,并且使用牛顿记法用函数符号上方的$n$个点表示其对时间的$n$阶导数,可以化为更常见的形式
        \begin{equation}\label{Newton's 2nd law}
            \bm{F} = m\ddot{\bm{r}}
        \end{equation}
        式中$F$为质点所受合外力,$m$为质点的惯性质量,$\ddot{\bm{r}}$为质点矢径对于时间的二阶导数。\par
        电磁场对电荷的作用可以由洛伦兹力方程得到
        \begin{equation}\label{Lorentz force}
            \bm{F} = q(\bm{E}+\bm{v}\times\bm{B})
        \end{equation}
        式中$\bm{v}$为电荷速度。\par
        电荷所受基本力中还包含万有引力,可以由万有引力公式的到
        \begin{equation}
            \bm{G} = G\frac{m_1m_2}{r^3}\bm{r}
        \end{equation}
        式中$G$为万有引力常数,$m_1,m_2$分别为两个质点的引力质量,大量实验证明,惯性质量与引力质量是等价的。$\bm{r}$为两个质点矢径之差,其模长$r$等于两个质点之间的距离。\par
        有了这几个方程,从理论上来讲我们可以解决一切的经典物理问题,实际上在19世纪末期科学家们大多是这么乐观地认为的,但还是有一些问题,就是开尔文所谓的“两朵乌云”,之后有了相对论和量子力学。但是对于宏观低速的物体经典力学仍然有相当高的精度。\par
        \subsection{数值分析原理}
        由以上公式可以看出电动力学最主要的计算是进行微积分计算和求解微分方程,接下来介绍数值计算中需要的数值分析原理。
            \subsubsection{导数}
            微积分中函数的导数定义为:
            \begin{equation}
                f'(x) = \lim_{h\to 0}\frac{f(x+h)-f(x)}{h}
            \end{equation}
            根据泰勒中值定理,若函数$f(x+h)$在$x$处具有$n$阶导数,在$U(x)$内,有
            \begin{equation}\label{Taylor}
                f(x+h) = \sum_i^nh^n\frac{f^{(n)}(x)}{n!}+R_n(x+h)
            \end{equation}
            式中$R_n(x+h)=o(h^n)$表示$h^n$的高阶无穷小,也被称为佩亚诺余项。此项为$n$次泰勒多项式来近似$f(x+h)$的误差,但其值无法确定,可以改为拉格朗日余项的形式来解决这一问题
            \begin{equation*}
                R_n(x+h) = \frac{f^{(n+1)}(\xi)}{(n+1)!}h^{n+1}
            \end{equation*}
            式中$\xi$为$(x,x+h)$中的某个值。\par
            如果仅取泰勒多项式\eqref{Taylor}式的三项得到
            \begin{equation}
                f(x+h) = f(x) + hf'(x) + \frac{h^2}{2}f''(\xi)
            \end{equation}
            经过移项后可以得到一阶导数的一种有限差分公式:
            \begin{equation}\label{Two-point forward difference}
                f'(x) = \frac{f(x+h)-f(x)}{h} - \frac{h}{2}f''(\xi)
            \end{equation}
            上式被称为二点前向差分公式。\par
            将\eqref{Two-point forward difference}式与导数的定义式相比较发现当$h\to 0$时就变成了导数的计算式。尽管在数值计算中不能取$h\to 0$的极限,但是当$h$非常小的时候我们认为
            \begin{equation}
                f'(x) \approx \frac{f(x+h)-f(x)}{h}
            \end{equation}
            通过对比我们发现\eqref{Two-point forward difference}式是用两点的割线代替了切线,由于误差$\frac{h}{2}f''(\xi)$与$h$成正比,因此\eqref{Two-point forward difference}式为近似一阶导数的一阶方法。一般地,如果误差为$O(h^n)$,那么我们称该公式为$n$阶方法。\par
            如果采用更高级的策略,可以得到不同的结果。取泰勒多项式\eqref{Taylor}式中的四项,并且写出$h=-h$时的泰勒多项式
            \begin{equation*}
                \begin{array}{l}
                    f(x+h) = f(x) + hf'(x) + \frac{h^2}{2}f''(x) + \frac{h^3}{6}f'''(\xi_1)\\
                    f(x-h) = f(x) - hf'(x) + \frac{h^2}{2}f''(x) - \frac{h^3}{6}f'''(\xi_2)
                \end{array}
            \end{equation*}
            将两式相减后可化简得到
            \begin{equation*}
                f'(x) = \frac{f(x+h)-f(x-h)}{2h} - \frac{h^2}{12}f'''(\xi_1) - \frac{h^2}{12}f'''(\xi_2)
            \end{equation*}
            应用推广介值定理
            \begin{equation}
                \sum_i^na_if(\xi) = \sum_i^n(a_if(x_i))
            \end{equation}
            可以得到三点中心差分公式
            \begin{equation*}
                f'(x) = \frac{f(x+h)-f(x-h)}{2h} - \frac{h^2}{6}f'''(\xi)
            \end{equation*}
            由误差项$\frac{h^2}{6}f'''(\xi)$可知三点中心差分公式是一个二阶公式。\par
            若取泰勒多项式\eqref{Taylor}式中的五项
            \begin{equation*}
                \begin{footnotesize}
                    \begin{array}{l}
                        f(x+h) = f(x) + hf'(x) + \frac{h^2}{2}f''(x) + \frac{h^3}{6}f'''(x) + \frac{h^4}{24}f'''(\xi)\\
                        \\
                        f(x-h) = f(x) - hf'(x) + \frac{h^2}{2}f''(x) - \frac{h^3}{6}f'''(x) + \frac{h^4}{24}f'''(\xi)
                    \end{array}
                \end{footnotesize}
            \end{equation*}
            将上式相加化简后得到二阶导数的三点中心差分公式:
            \begin{equation}\label{three-point center difference_2}
                \begin{split}
                    f''(x) = &\frac{f(x+h)-2f(x)+f(x-h)}{h^2}\\
                    &-\frac{h^2}{12}f^{(4)}(\xi)
                \end{split}
            \end{equation}
            上式同样是一个二阶公式。\par
            我们可以外推到$n$阶公式,这种方式也被称为理查德森外推
            \begin{equation}
                Q\approx\frac{2^nF(h/2)-F(h)}{2^n-1}
            \end{equation}
            相对于$F(h)$,理查德森外推给出对$Q$更高阶的近似。由此我们可以从二阶的三点中心差分公式外推到四阶公式,这将是我们计算中主要使用的方法。
            \subsubsection{定积分}
            若我们将区间$[a,b]$分成$n$个小区间,定义$\lambda = max\{\Delta x_i\}$,定积分的定义如下:
            \begin{equation}
                \int_{a}^{b}f(x){\rm d}x = \lim_{\lambda\to 0}\sum_{i=0}^{n}f(\xi_i)\Delta x_i
            \end{equation}\par
            根据微积分的原理,我们可以很轻松地计算出多项式的积分值,因此我们用插值函数来代替待求积分的函数,利用这种方式计算定积分的公式被称为牛顿-科特斯公式。首先考虑函数两个端点的插值,显然应该为一条直线,用直线的积分来近似函数的积分实际上就是利用梯形近似曲边梯形,因此我们得到梯形公式
            \begin{equation}
                \int_{x_0}^{x_1} f(x){\rm d}x = \frac{h}{2}(y_0+y_1)-\frac{h^3}{12}f''(\xi)
            \end{equation}
            其中$h = x_1 - x_0$$\xi$在$x_0$到$x_1$之间。\par
            我们加上区间中点进行插值,可以得到一条抛物线,于是有辛普森公式
            \begin{equation}
                \int_{x_0}^{x_2} f(x){\rm d}x = \frac{h}{3}(y_0+4y_1+y_2)-\frac{h^5}{90}f''(\xi)
            \end{equation}
            其中$h = x_2 - x_1 = x_1 - x_0$$\xi$在$x_0$到$x_2$之间
            如果我们取更多点,会得到更精确的积分数值计算式,但是我们也可以通过将原本的积分区间细分,再对每一个区间进行计算,这被称为复合牛顿-科特斯公式,例如复合梯形公式
            \begin{equation}
                \begin{split}
                    \int_a^b f(x){\rm d}x = &\frac{h}{2}(y_0+y_m+2\sum_{i=1}^{m-1}yx_i)\\
                    &-\frac{(b-a)h^2}{12}f''(\xi)
                \end{split}
            \end{equation}
            其中$h = \frac{(b-a)}{m}$$\xi$在$a$到$b$之间,如果步长趋近于无穷小,那么数值积分值将会趋近真实值。\par
            对辛普森公式进行同样的操作,可以得到复合辛普森公式
            \begin{equation}
                \begin{split}
                    \int_a^b f(x){\rm d}x = &\frac{h}{3}(y_0+y_{2m}+4\sum_{i=1}^{m}y_{2i-1}\\
                    &+2\sum_{i=1}^{m-1}y_{2i})-\frac{(b-a)h^4}{180}f^{(4)}(c)
                \end{split}
            \end{equation}
            其中$c$在$a$到$b$之间。\par
            利用理查德森外推,我们也可以将牛顿-科特斯公式外推到$n$阶公式,也被称为龙贝格积分公式
            \begin{equation}
                R_{11} = \frac{h}{2}(f(a)+f(b))
            \end{equation}
            \begin{equation}\label{Romberg1}
                R_{j1} = \frac{1}{2}R_{i-1,1} + h_j\sum_{i=1}^{2^{j-2}}f(a+(2i-1)h_j)
            \end{equation}
            \begin{equation}\label{Romberg2}
                R_{jk} = \frac{4^{k-1}R_{j,k-1}-R_{j-1,k-1}}{4^{k-1}-1}
            \end{equation}
            只需要不断循环\eqref{Romberg1}和\eqref{Romberg2}便可以生成一个下三角矩阵,$R_{jj}$的数值对应$2j$阶近似。龙贝格积分可以计算$R_{jj}$与$R_{jj-1}$的差值来控制误差,并且可以有效地利用之前的计算结果,本次数值实验的积分将采用龙贝格积分。
            \subsubsection{常微分方程}
            牛顿运动方程是一个二阶常微分方程(Ordinay Differential Equation,简称ODE),只要我们能求解出运动轨迹的方程,就能计算出质点任意时刻的位置、速度、加速度等信息。求解微分方程可以看作在斜率场内顺着斜率画出一条轨迹,但通常这样的轨迹有无数条,因此我们需要一个初始值来确定一条轨迹,这类问题称为初值问题(IVP)。\par
            我们同样使用有限差分法,利用折线代替曲线,只要折线足够短且误差积累不多,通常与真实曲线误差不大,设一阶微分方程:
            \begin{equation}
                \left\{ \begin{array}{l}
                    \frac{dy}{dt} = f(t,y)\\
                    f(t_0) = y_0
                \end{array}\right.
            \end{equation}
            则有欧拉方法:
            \begin{equation}
                \begin{split}
                    \omega_0 =& y_0\\
                    \omega_{i+1} =& \omega_i + hf(t_i,\omega_i)
                \end{split}
            \end{equation}
            $\omega_i$是欧拉方法估计的$y_i$值,可以证明欧拉方法是一阶方法,并且函数单调时误差会积累。\par
            只需要对欧拉方法的公式做一个小调整,就可以对精度有很大的提高:
            \begin{equation}
                \begin{split}
                    \omega_0 =& y_0\\
                    \omega_{i+1} =& \omega_i + \frac{h}{2}(f(t_i,\omega_i)\\
                    &+f(t_i+h,\omega_i+hf(ti,\omega_i)))
                \end{split}
            \end{equation}
            这种方法被称为梯形方法。\par
            由于欧拉方法和梯形方法都可以用旧估计值通过显式公式确定,因此被称为显式方法。也有诸如后向欧拉方法和隐式梯形法等隐式方法,对于一个具体的微分方程可以显化成一个可以迭代的显式表达式,目前不需要使用这些方法,暂时不提。\par
            龙格-库塔方法是一组ODE求解器,包含欧拉方法和梯形方法,以及更复杂的高阶方法。中点方法类似梯形方法,不同的是梯形方法使用区间的右端使用欧拉方法求值,然后与左端取平均值作为斜率;而中点法则使用区间中点的斜率来代替。最常见的龙格-库塔方法是四阶方法(RK4):
            \begin{equation}
                \omega_{i+1} = \omega_i + \frac{h}{6}(s_1+2s_2+2s_3+s_4)
            \end{equation}
            其中
            \begin{equation*}
                \begin{array}{l}
                    s_1 = f(t_i,\omega_i)\\
                    s_2 = f(t_i+\frac{h}{2},\omega_i+\frac{h}{2}s_1)\\
                    s_3 = f(t_i+\frac{h}{2},\omega_i+\frac{h}{2}s_2)\\
                    s_4 = f(t_i+h,\omega_i+hs_3)
                \end{array}
            \end{equation*}
            这种方法本身简单并且易于编程实现,因此非常流行。\par
            \subsubsection{偏微分方程}
            麦克斯韦方程组是一个偏微分方程(Partial Differential Equation,简称PDE)组,和常微分方程有很多不一样的地方。\par
            若以两个变量的偏微分方程举例,二阶线性偏微分方程组可以表示为:
            \begin{equation}\label{PDE_2D}
                    Au_{xx} + Bu_{xy} + Cu_{yy} + F(u_x,u_y,u,x,y) = 0
            \end{equation}
            其中偏导数使用下标$x$和$y$表示对应的独立变量,$u$表示解。在热方程或波动方程中有一个变量表示时间,我们倾向于称独立变量为$x$和$t$。实际物理问题应该看作一个四维的偏微分方程,四个独立变量分别为$x,y,z,t$,但是如果要绘制成图最多绘制成三维图像,并且用不同颜色来表示该点的数值大小,并且这样绘制出来的图并不适合展示在平面上。因此之后我们求解的偏微分方程至多只有两个独立变量,尽管更高阶的解法同样存在。\par
            根据\eqref{PDE_2D}中主导阶项,可以按解的性质完全不同而分类如下:\par
            1)$B^2 - 4AC = 0$,抛物线方程\par
            2)$B^2 - 4AC > 0$,双曲线方程\par
            3)$B^2 - 4AC < 0$,椭圆方程\par
            抛物线方程的形式通常被称为热方程或扩散方程:
            \begin{equation}
                \frac{\partial u}{\partial t} = D\nabla^2 u
            \end{equation}
            $D$为扩散系数,通常是常数。扩散方程可以直接由连续性方程导出。\par
            双曲线方程的形式通常被称为波动方程:
            \begin{equation}
                \frac{\partial^2u}{\partial t^2} = c^2\nabla^2 u
            \end{equation}
            $c$为波速。电磁波的波动方程可以直接由麦克斯韦方程组直接推导出。\par
            椭圆方程的形式通常被称为泊松方程:
            \begin{equation}
                \nabla^2 u = f
            \end{equation}
            特别地,当$f=0$,泊松方程退化成拉普拉斯方程,拉普拉斯方程的解称为调和函数。\par
            泊松方程在物理中有许多应用,也是本次作业之一,其解表示势能。电场$\bm{E}$是电势$\varphi$的负梯度:
            \begin{equation}
                \bm{E} = -\nabla\varphi
            \end{equation}
            再代入麦克斯韦方程组的电场散度式可以得到势能$\varphi$的泊松方程:
            \begin{equation}
                \partial^2\varphi = -\frac{\rho}{\varepsilon}
            \end{equation}
            重力势能也可以表示为与密度有关的泊松方程,稳态的热分布则可以表示为拉普拉斯方程。\par
            根据作业要求只介绍椭圆方程的PDE求解器。对于椭圆方程,主要是有限差分方法和有限元方法两种方法。有限元分析使用非常广泛,诸如计算流体力学等都有应用,除此之外还有有限差分方法。\par
            有限差分方法在之前已经多次使用,实际上对于偏微分方程同样适用,而对于二维的椭圆方程,我们就需要在两个方向进行差分。\par
            假设我们将要求解矩形区域$[x_l.x_r]\times[y_b,y_t]$上的泊松方程$\nabla^2u=f$,狄利克雷边界条件如下
            \begin{equation*}
                \begin{split}
                    u(x,y_b) &= g_1(x)\\
                    u(x,y_t) &= g_2(x)\\
                    u(x_l,y) &= g_3(y)\\
                    u(x_r,y) &= g_4(y)
                \end{split}
            \end{equation*}
            将求解区域离散为矩形的点阵网格,在水平方向使用$M = m-1$步,在竖直方向使用$N = n-1$步,在$x$和$y$方向的网格大小分别是$h=(x_r-x_l)/M$和$k=(y_t-y_b)/N$,使用差商近似导数。\eqref{three-point center difference_2}可用于拉普拉斯算子的二阶导数,于是泊松方程具有如下的有限差分形式
            \begin{equation}
                \begin{split}
                    \frac{\omega_{i-1,j}-2\omega_{ij}+\omega_{i+1,j}}{h^2}&\\
                    +\frac{\omega_{i,j-1}-2\omega_{ij}+\omega_{i,j+1}}{k^2} &= f(x_i,y_i)                   
                \end{split}
            \end{equation}
            其中$x_i=x_l+(i-1)h,y_j=y_b+(j-1)k,\omega_{ij}\approx u(x_i,y_j)$并且$1\le i\le m,1\le j\le n$。\par
            由于$\omega_{ij}$的方程是线性的,我们可以矩阵求解$mn$个未知变量,但是这带来了记录的问题,因此我们将双引索的未知变量重新标记为线性顺序,设$v_{i+(j-1)m} = \omega_{ij}$,然后构造矩阵$\bm{A}$及向量$b$满足$\bm{A}v=b$,并求解$v$,然后再变换回原矩形的解$\omega$。由于$v$是长度为$mn$的向量,$\bm{A}$为$mn\times mn$矩阵,每一个格点都有自己对应的线性方程。\par
            设$\bm{A}_{pq}$是$\bm{A}v=b$第$p$个方程的第$q$个线性系数,则
            \begin{equation}
                \begin{split}
                    \bm{A}_{i+(j-1)m,i+(j-1)m} &= -\frac{1}{h^2}-\frac{1}{k^2}\\
                    \bm{A}_{i+(j-1)m,i+1+(j-1)m} &= \frac{1}{h^2}\\
                    \bm{A}_{i+(j-1)m,i-1+(j-1)m} &= \frac{1}{h^2}\\
                    \bm{A}_{i+(j-1)m,i+jm} &= \frac{1}{k^2}\\
                    \bm{A}_{i+(j-1)m,i+(j-2)m} &= \frac{1}{k^2}\\
                    b_{i+(j-1)m} &= f(x_i,y_j)
                \end{split}
            \end{equation}
            其中$1<i<m,1<j<n$。\par
            狄利克雷条件表示为
            \begin{equation}
                \begin{split}
                    \bm{A}_{i,i} = 1&,b_{i} = g_1(x_i)\\
                    \bm{A}_{i+(n-1)m,i+1+(n-1)m} = 1&,b_{i+(n-1)m} = g_2(x_i)\\
                    \bm{A}_{1+(j-1)m,1-1+(j-1)m} = 1&,b_{1+(j-1)m} = g_4(y_j)\\
                    \bm{A}_{jm,(j+1)m} = 1&,b_{jm} = g_4(y_j)
                \end{split}
            \end{equation}
            其中$1\le i\le m,1\le j\le n$。\par
            $\bm{A}$和$b$剩下的元素均为0,只需要将这个线性方程组求解就可以得到泊松方程的解,求解线性方程组可以使用高斯消元法或者其他迭代方法。
    \section{数值计算}
    根据以上原理,我们可以通过计算机编程来实现数值分析的算法,目前家用计算机计算速度一般在2GHz$\sim$4GHz,即每秒可进行$10^9$次运算,超级计算机更可以达到每秒计算$10^{17}$次的数量级,远超于人类,广泛应用于各类科学研究。
        \subsection{程序与算法实现}
        目前有许多数值分析软件供科研人员使用,常见的有MATLAB、Mathmatica、Maple等,以上均为开发完成的商业软件。除此之外也可以使用计算机高级语言编写程序,如Python、Julia、Java、R、C++、Fortran等,这些基本都是免费并开源的编程语言,相比于专业商业软件要便宜。\par
        本次作业均由Python完成,Python有众多的第三方库,在机器学习和数值分析中非常受欢迎,可以在官网免费下载或者通过pip命令在命令行中直接下载安装,但是pip默认使用的网站位于国外,如果速度较慢可以改用国内镜像。常用第三方库中,Numpy和Scipy比较适合用于数值分析,Sympy可以进行代数计算,MatPlotLib可以提供和MATLAB相似的绘图功能。\par
            \subsubsection{数值微积分}
            首先使用有限差分法离散库仑定律
            \lstinputlisting[breaklines,language=Python]{Coulomb.py}
            数值微分主要使用的是四点中心差分公式,以此为基础利用Numpy实现了数值微分、偏微分、梯度运算、散度运算,其中偏微分仅做了二维情况,根据原理可以很轻松地推广到三维,但需要注意的是一旦维度提高,算法的时间复杂度也会增加
            \lstinputlisting[breaklines,language=Python]{numerical_differential.py}
            数值积分主要使用牛顿-科特斯公式,以辛普森公式为主,也包含二重积分的辛普森公式、开区间的牛顿-科特斯公式、龙贝格积分公式、自适应积分,主要使用Numpy实现
            \lstinputlisting[breaklines,language=Python]{numerical_integral.py}
            \subsubsection{常微分方程}
            ODE求解器有许多种,其中四阶龙格-库塔法较为常见,也便于程序实现,以下是利用Numpy实现的三种ODE求解器
            \lstinputlisting[breaklines,language=Python]{ordinary_differential_equation.py}
            \subsubsection{偏微分方程}
            利用有限差分方法可以实现定解条件为矩形区域上的狄利克雷边界条件的泊松方程求解器,同样利用Numpy实现
            \lstinputlisting[breaklines,language=Python]{partial_differential_equations.py}
        \subsection{数值实验}
        接下来利用我们写好的程序来进行数值模拟,再与解析解进行对比,就可以大概知道我们算法的精确性,如果足够精确则可以用来进行一些非解析解的计算。
            \subsubsection{带电粒子在电磁场中的运动}
            带电粒子在电磁场中受到洛伦兹力的作用,联立\eqref{Newton's 2nd law}和\eqref{Lorentz force}可得到带电粒子的运动方程
            \begin{equation}
                m\ddot{r} = q(\bm{E} + \dot{r}\times \bm{B})
            \end{equation}
            这是一个二阶常微分方程,改写为一阶常微分方程组的形式
            \begin{equation}
                \left\{\begin{array}{l}
                    \dot{r} = v\\
                    \dot{v} = \frac{q}{m}(\bm{E} + \bm{v}\times \bm{B})
                \end{array}\right.
            \end{equation}
            可以使用ODE求解器RK4来求解该常微分方程组,首先是恒定纯磁场的情况,$\bm{F} = q\bm{v}\times\bm{B}$向量外积得到的新向量与原来的两个向量都垂直,因此洛伦兹力垂直于速度方向不做功,运动轨迹应该是一条直线或者圆,这取决于速度方向是否与磁场方向平行
            再考虑电磁场同时存在且恒定的情况。最简单的情况是电场方向与磁场方向平行,这时电场力和磁场力的作用相互独立,运动轨迹应当是一个半径不变的螺旋线
            其他的情况无解析解,但可以进行数值模拟
            \subsubsection{电荷、电场与电势的计算}
            已知电荷分布可以利用库仑定律求得电场强度,真空中点电荷的电场强度由\eqref{Coulomb_1}得到,
    \section{结论}
\end{multicols}
\end{document}